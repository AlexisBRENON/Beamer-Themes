\documentclass[aspectratio=169]{beamer}

\usetheme{ab2015}
\usepackage{minted}
\usepackage{tikz}


\title{Title : A beamer theme showcase}
\subtitle{Subtitle : Show main lines of a beamer theme}
\date{\today}
\author{Alexis BRENON}
\institute{Institute or miscellaneous information}

\twitter{AlexisBRENON}
\website{alexisbrenon.github.io}

\begin{document}

\openingpage

\begin{frame}[fragile]
  \frametitle{Basic Frame}

  This is a basic frame. Nothing really fancy, except this monospaced text below.
  Enable a theme using

  \begin{minted}{latex}
    \documentclass{beamer}
    \usetheme{themename}
  \end{minted}
\end{frame}

\part{Typography}
\partpage

\begin{frame}[fragile]
  \frametitle{Typography}
  
  The theme provides sensible defaults to \emph{emphasize} text,
  \alert{alert} parts or show \textbf{bold} results.
\end{frame}

\begin{frame}{Lists}{Items}
  \begin{columns}[onlytextwidth]
    \column{0.5\textwidth}
      Items
      \begin{itemize}
        \item Milk
        \item Juices \begin{itemize}
            \item Apple
            \item Orange
        \end{itemize}
        \item Potatoes
      \end{itemize}

    \column{0.5\textwidth}
      Enumerations
      \begin{enumerate}
        \item First, \item Second and \item Last.
      \end{enumerate}
  \end{columns}
\end{frame}

\begin{frame}{Lists}{Descriptions}
  \begin{description}
    \item[PowerPoint] Meeh.
    \item[Beamer] Yeeeha.
  \end{description}
\end{frame}

\begin{frame}{Quotes}
  \begin{quote}
    Veni, Vidi, Vici
  \end{quote}
\end{frame}

\part{Extended features}
\partpage

\section{Animation}

\begin{frame}{Animation}
  \begin{itemize}[<+- | alert@+>]
    \item \alert<4>{This is\only<4>{ really} important}
    \item Now this
    \item And now this
  \end{itemize}
\end{frame}

\section{Blocks}

\begin{frame}{Blocks}
  \begin{block}{This is a normal block title}
	And its text
  \end{block}
  
  \begin{alertblock}{This is an alert block title}
	And its alert text
  \end{alertblock}
  
  \begin{exampleblock}{This is an example block title}
	And its example text
  \end{exampleblock}
\end{frame}

\section{Figures and math}

%\begin{frame}{Figures}
%  \begin{figure}
%    \newcounter{density}
%    \setcounter{density}{20}
%    \begin{tikzpicture}
%      \def\couleur{mLightBrown}
%      \path[coordinate] (0,0)  coordinate(A)
%                  ++( 90:5cm) coordinate(B)
%                  ++(0:5cm) coordinate(C)
%                  ++(-90:5cm) coordinate(D);
%      \draw[fill=\couleur!\thedensity] (A) -- (B) -- (C) --(D) -- cycle;
%      \foreach \x in {1,...,40}{%
%          \pgfmathsetcounter{density}{\thedensity+20}
%          \setcounter{density}{\thedensity}
%          \path[coordinate] coordinate(X) at (A){};
%          \path[coordinate] (A) -- (B) coordinate[pos=.10](A)
%                              -- (C) coordinate[pos=.10](B)
%                              -- (D) coordinate[pos=.10](C)
%                              -- (X) coordinate[pos=.10](D);
%          \draw[fill=\couleur!\thedensity] (A)--(B)--(C)-- (D) -- cycle;
%      }
%    \end{tikzpicture}
%    \caption{Rotated square from
%    \href{http://www.texample.net/tikz/examples/rotated-polygons/}{texample.net}.}
%  \end{figure}
%\end{frame}

%\begin{frame}{Tables}
%  \begin{table}
%    \caption{Largest cities in the world (source: Wikipedia)}
%    \begin{tabular}{lr}
%      \toprule
%      City & Population\\
%      \midrule
%      Mexico City & 20,116,842\\
%      Shanghai & 19,210,000\\
%      Peking & 15,796,450\\
%      Istanbul & 14,160,467\\
%      \bottomrule
%    \end{tabular}
%  \end{table}
%\end{frame}

\begin{frame}{Math}
  \begin{equation*}
    e = \lim_{n\to \infty} \left(1 + \frac{1}{n}\right)^n
  \end{equation*}
\end{frame}

%\begin{frame}{Line plots}
%  \begin{figure}
%    \begin{tikzpicture}
%      \begin{axis}[
%        mlineplot,
%        width=0.9\textwidth,
%        height=6cm,
%      ]
%
%        \addplot {sin(deg(x))};
%        \addplot+[samples=100] {sin(deg(2*x))};
%
%      \end{axis}
%    \end{tikzpicture}
%  \end{figure}
%\end{frame}

%\begin{frame}{Bar charts}
%  \begin{figure}
%    \begin{tikzpicture}
%      \begin{axis}[
%        mbarplot,
%        xlabel={Foo},
%        ylabel={Bar},
%        width=0.9\textwidth,
%        height=6cm,
%      ]
%
%      \addplot plot coordinates {(1, 20) (2, 25) (3, 22.4) (4, 12.4)};
%      \addplot plot coordinates {(1, 18) (2, 24) (3, 23.5) (4, 13.2)};
%      \addplot plot coordinates {(1, 10) (2, 19) (3, 25) (4, 15.2)};
%
%      \legend{lorem, ipsum, dolor}
%
%      \end{axis}
%    \end{tikzpicture}
%  \end{figure}
%\end{frame}

\closingpage{Questions?}

\end{document}
